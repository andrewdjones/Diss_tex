\centerline{Abstract}

\centerline{Harmony and Statistical Temporality: Toward Jazz Syntax from Corpus Analytics}

\centerline{Andrew Daniel Jones}

\centerline{2017}

The study of jazz harmony treats the configuration and concatenation of chords within a phrase, performance, tune, style, or genre.  Harmonic analysts typically seek to segment musical surfaces into collections of notes sounded (nearly) simultaneously, and these chords receive labels imbricated in a representation system capturing (and partially defining) their structural properties and functional uses.  Since at least the 1970s, music theorists have drawn on paradigms from linguistics to describe patterns and norms in the deployment of the resulting chord objects.  If jazz (and music generally) is language-like, then the normative or allowable orderings of chords in time might bear some structural resemblance to the \emph{syntax} of well-formed sentences.  This dissertation questions and reframes the concept of harmonic syntax, drawing on a corpus of jazz piano performances and elementary machine learning to construct data-driven chord categories which generalize across traditional notions of harmonic progression and function.

In Chapter 1, I argue that traditional, roman-numeral type data representations cannot provide a ground for assessing extant theories of syntax, linguistic or otherwise.  An examination of the labeling and parsing procedures employed by a range of jazz theorists suggests that content-based chord labeling, which implicitly or explicitly identifies chord similarity based on the pitch content of individual verticalities, both cuts across and pre-supposes assumptions regarding the contextual behavior of the resulting chord categories.  After placing these latent syntactic theories in dialog with Chomskyan linguistics and recent computational work, I suggest that comparatively content-free chord labeling agnostic about the structure and deployment of the chords represented can underpin statistical claims regarding a surface-oriented local harmonic syntax.

Chapter 2 pursues such chord labeling in the context of voicings and locally transposed scale-degree sets observed in the Yale Jazz MIDI Piano corpus (YJaMP).  Relaxing chord labeling assumptions regarding rooted, tertian-stack, diatonic chord structures extends chord labels to a wide array of harmonic objects.  I frame the process of turning musical surfaces into harmonic objects with semiotic terms taken from Paul Kockelman, and I claim that algorithmic parsing of MIDI performance data can produce chord labels uniquely suited to the contextual chord categorization of Chapters 3 and 4.  Algorithmic processes here and elsewhere in the dissertation do not explain how humans perceive or conceptualize jazz piano syntax; rather, they provide a minimally-circular way to specify a meaning for chord labeling claims that is temporally sensitive and syntactically productive.

Chapter 3 provides a framework for investigating the temporal behavior of the generalized chord objects identified in Chapter 2.  Taking the traditional $ii$ chord as a case study, I challenge traditional notions of ``progressions" as bigram adjacencies.  Attending to the probabilistic statistics for chords following $ii$ at time delays of up to five seconds, I suggest that the contextual behavior of voice-leading neighbors (like other $ii$ chords) and syntactic progression destinations (like $V$ chords) can provide temporal probability templates against which other chord behaviors may be measured.  Drawing on time series analytical approaches common in machine learning, I show that principal component analysis (PCA) allows the automated extraction of these templates from the statistics for $ii$ or any other chord.  The resulting principal components provide a computational translation of analytical intuitions regarding phonetic-scale and syntactic-scale harmonic progressions.

Chapter 4 generalizes the PCA-reduced temporal statistics for $ii$ to the full set of YJaMP scale-degree sets, introducing a quantitative basis for comparing the behavior of arbitrary chord structures across multiple time scales.  With time series statistics embedded in the chord representation, simple metrics and machine learning methods then permit surprisingly sensitive chord categorization schemes.  I show that agglomerative hierarchical clustering reproduces human analytical expectations regarding syntactic category formation, and na\"{i}ve supervised classifiers like k-nearest neighbors can place new or lower-probability chords into the resulting framework with low computational overhead.

Chapter 5 places the categorization of Chapters 2-4 back in the context of formalist music theories, drawing on critiques from Richard Wollheim and Paul Kockelman to indicate that the data analytical pipeline in these pages avoids many common kinds of circularity while also limiting its human interpretability.  I mean the representation and classification scheme advocated here to serve as a starting point, not an ending, and I suggest several modes of algorithmic-human semiotic communication which might yield productive observations regarding jazz syntax.  In a way, I situate syntactic descriptions of temporal progressions as emerging from the relations between agents suited to different tasks -- and I hope to spur harmonic claims more accountable to the means of their own production than to the assumed properties of pre-existing harmonic objects.